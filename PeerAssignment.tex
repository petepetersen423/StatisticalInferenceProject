\PassOptionsToPackage{unicode=true}{hyperref} % options for packages loaded elsewhere
\PassOptionsToPackage{hyphens}{url}
%
\documentclass[]{article}
\usepackage{lmodern}
\usepackage{amssymb,amsmath}
\usepackage{ifxetex,ifluatex}
\usepackage{fixltx2e} % provides \textsubscript
\ifnum 0\ifxetex 1\fi\ifluatex 1\fi=0 % if pdftex
  \usepackage[T1]{fontenc}
  \usepackage[utf8]{inputenc}
  \usepackage{textcomp} % provides euro and other symbols
\else % if luatex or xelatex
  \usepackage{unicode-math}
  \defaultfontfeatures{Ligatures=TeX,Scale=MatchLowercase}
\fi
% use upquote if available, for straight quotes in verbatim environments
\IfFileExists{upquote.sty}{\usepackage{upquote}}{}
% use microtype if available
\IfFileExists{microtype.sty}{%
\usepackage[]{microtype}
\UseMicrotypeSet[protrusion]{basicmath} % disable protrusion for tt fonts
}{}
\IfFileExists{parskip.sty}{%
\usepackage{parskip}
}{% else
\setlength{\parindent}{0pt}
\setlength{\parskip}{6pt plus 2pt minus 1pt}
}
\usepackage{hyperref}
\hypersetup{
            pdftitle={Statistical Inference - Peer Graded Assignment},
            pdfauthor={Pete Petersen III},
            pdfborder={0 0 0},
            breaklinks=true}
\urlstyle{same}  % don't use monospace font for urls
\usepackage[margin=1in]{geometry}
\usepackage{color}
\usepackage{fancyvrb}
\newcommand{\VerbBar}{|}
\newcommand{\VERB}{\Verb[commandchars=\\\{\}]}
\DefineVerbatimEnvironment{Highlighting}{Verbatim}{commandchars=\\\{\}}
% Add ',fontsize=\small' for more characters per line
\usepackage{framed}
\definecolor{shadecolor}{RGB}{248,248,248}
\newenvironment{Shaded}{\begin{snugshade}}{\end{snugshade}}
\newcommand{\AlertTok}[1]{\textcolor[rgb]{0.94,0.16,0.16}{#1}}
\newcommand{\AnnotationTok}[1]{\textcolor[rgb]{0.56,0.35,0.01}{\textbf{\textit{#1}}}}
\newcommand{\AttributeTok}[1]{\textcolor[rgb]{0.77,0.63,0.00}{#1}}
\newcommand{\BaseNTok}[1]{\textcolor[rgb]{0.00,0.00,0.81}{#1}}
\newcommand{\BuiltInTok}[1]{#1}
\newcommand{\CharTok}[1]{\textcolor[rgb]{0.31,0.60,0.02}{#1}}
\newcommand{\CommentTok}[1]{\textcolor[rgb]{0.56,0.35,0.01}{\textit{#1}}}
\newcommand{\CommentVarTok}[1]{\textcolor[rgb]{0.56,0.35,0.01}{\textbf{\textit{#1}}}}
\newcommand{\ConstantTok}[1]{\textcolor[rgb]{0.00,0.00,0.00}{#1}}
\newcommand{\ControlFlowTok}[1]{\textcolor[rgb]{0.13,0.29,0.53}{\textbf{#1}}}
\newcommand{\DataTypeTok}[1]{\textcolor[rgb]{0.13,0.29,0.53}{#1}}
\newcommand{\DecValTok}[1]{\textcolor[rgb]{0.00,0.00,0.81}{#1}}
\newcommand{\DocumentationTok}[1]{\textcolor[rgb]{0.56,0.35,0.01}{\textbf{\textit{#1}}}}
\newcommand{\ErrorTok}[1]{\textcolor[rgb]{0.64,0.00,0.00}{\textbf{#1}}}
\newcommand{\ExtensionTok}[1]{#1}
\newcommand{\FloatTok}[1]{\textcolor[rgb]{0.00,0.00,0.81}{#1}}
\newcommand{\FunctionTok}[1]{\textcolor[rgb]{0.00,0.00,0.00}{#1}}
\newcommand{\ImportTok}[1]{#1}
\newcommand{\InformationTok}[1]{\textcolor[rgb]{0.56,0.35,0.01}{\textbf{\textit{#1}}}}
\newcommand{\KeywordTok}[1]{\textcolor[rgb]{0.13,0.29,0.53}{\textbf{#1}}}
\newcommand{\NormalTok}[1]{#1}
\newcommand{\OperatorTok}[1]{\textcolor[rgb]{0.81,0.36,0.00}{\textbf{#1}}}
\newcommand{\OtherTok}[1]{\textcolor[rgb]{0.56,0.35,0.01}{#1}}
\newcommand{\PreprocessorTok}[1]{\textcolor[rgb]{0.56,0.35,0.01}{\textit{#1}}}
\newcommand{\RegionMarkerTok}[1]{#1}
\newcommand{\SpecialCharTok}[1]{\textcolor[rgb]{0.00,0.00,0.00}{#1}}
\newcommand{\SpecialStringTok}[1]{\textcolor[rgb]{0.31,0.60,0.02}{#1}}
\newcommand{\StringTok}[1]{\textcolor[rgb]{0.31,0.60,0.02}{#1}}
\newcommand{\VariableTok}[1]{\textcolor[rgb]{0.00,0.00,0.00}{#1}}
\newcommand{\VerbatimStringTok}[1]{\textcolor[rgb]{0.31,0.60,0.02}{#1}}
\newcommand{\WarningTok}[1]{\textcolor[rgb]{0.56,0.35,0.01}{\textbf{\textit{#1}}}}
\usepackage{longtable,booktabs}
% Fix footnotes in tables (requires footnote package)
\IfFileExists{footnote.sty}{\usepackage{footnote}\makesavenoteenv{longtable}}{}
\usepackage{graphicx,grffile}
\makeatletter
\def\maxwidth{\ifdim\Gin@nat@width>\linewidth\linewidth\else\Gin@nat@width\fi}
\def\maxheight{\ifdim\Gin@nat@height>\textheight\textheight\else\Gin@nat@height\fi}
\makeatother
% Scale images if necessary, so that they will not overflow the page
% margins by default, and it is still possible to overwrite the defaults
% using explicit options in \includegraphics[width, height, ...]{}
\setkeys{Gin}{width=\maxwidth,height=\maxheight,keepaspectratio}
\setlength{\emergencystretch}{3em}  % prevent overfull lines
\providecommand{\tightlist}{%
  \setlength{\itemsep}{0pt}\setlength{\parskip}{0pt}}
\setcounter{secnumdepth}{0}
% Redefines (sub)paragraphs to behave more like sections
\ifx\paragraph\undefined\else
\let\oldparagraph\paragraph
\renewcommand{\paragraph}[1]{\oldparagraph{#1}\mbox{}}
\fi
\ifx\subparagraph\undefined\else
\let\oldsubparagraph\subparagraph
\renewcommand{\subparagraph}[1]{\oldsubparagraph{#1}\mbox{}}
\fi

% set default figure placement to htbp
\makeatletter
\def\fps@figure{htbp}
\makeatother


\title{Statistical Inference - Peer Graded Assignment}
\author{Pete Petersen III}
\date{6/20/2020}

\begin{document}
\maketitle

\hypertarget{part-1-simulation-exercise}{%
\section{Part 1: Simulation Exercise}\label{part-1-simulation-exercise}}

\hypertarget{overview}{%
\paragraph{Overview:}\label{overview}}

In this paper we demonstrate that the Central Limit Therom is in
congruence with theoretical normal distributions. In order to accomplish
this we will generate a 1000 samples of lenf 40, We will then take the
means of each of these 1000 samples. This will show that the density of
the means is very sumilar to the theoretic density. The exponential
distribution will be simulated in R with rexp(n,lambda) where lambda is
the rate parameter. rexp uses Ahrens, J. H. and Dieter, U. (1972).
Computer methods for sampling from the exponential and normal
distributions. Communications of the ACM, 15, 873--882.

\hypertarget{simulations}{%
\paragraph{Simulations:}\label{simulations}}

The mean of exponential distribution and the standard deviation are both
1/lambda where lambda = 0.2, and distribution of averages of 40
exponentials and will perform 1000 simulations.

\hypertarget{the-exponential-distribution}{%
\paragraph{The Exponential
Distribution:}\label{the-exponential-distribution}}

\begin{verbatim}
Density, distribution function, quantile function and random generation for the exponential distribution with rate (i.e., mean 1/rate).

 Usage
    dexp(x, rate = 1, log = FALSE)
    pexp(q, rate = 1, lower.tail = TRUE, log.p = FALSE)
    qexp(p, rate = 1, lower.tail = TRUE, log.p = FALSE)
    rexp(n, rate = 1)
   
   Arguments:
   
    x, q -  vector of quantiles.
    p   - vector of probabilities.
    n - number of observations. If length(n) > 1, the length is taken to be the number required.
    rate - vector of rates.
    log, log.p - logical; if TRUE, probabilities p are given as log(p).
    lower.tail - logical; if TRUE (default), probabilities are P[X ≤ x], otherwise, P[X > x].
\end{verbatim}

\begin{Shaded}
\begin{Highlighting}[]
 \CommentTok{# Set the initial simulations paremeters wher lamda is the the rate parameter in exponential distrubution.}
\NormalTok{  lambda <-}\StringTok{ }\FloatTok{0.2}

\CommentTok{# Set the seed for the random number generator for reproducabkility}
  \KeywordTok{set.seed}\NormalTok{(}\DecValTok{11}\NormalTok{)}

  \CommentTok{# Set the number of observations. If length(n) > 1, the length is taken to be the number required}
\NormalTok{  n <-}\StringTok{ }\DecValTok{40}

  \CommentTok{# Iterate  a thousand times of the exponential distribution where length = 40 and lamda = .2  THis will result in  a 40 row by 1000 column array.}
\NormalTok{  simulation <-}\StringTok{ }\KeywordTok{replicate}\NormalTok{(}\DecValTok{1000}\NormalTok{, }\KeywordTok{rexp}\NormalTok{(n, lambda))}
  
\CommentTok{# we use the apply to take means over the simulation distribution columns and arrive ar 1 mean per colimn.}
\NormalTok{  mean_simulation <-}\StringTok{ }\KeywordTok{apply}\NormalTok{(simulation, }\DecValTok{2}\NormalTok{, mean)}
\end{Highlighting}
\end{Shaded}

Sample Mean versus Theoretical Mean: Include figures with titles. In the
figures, highlight the means you are comparing. Include text that
explains the figures and what is shown on them, and provides appropriate
numbers.

\begin{Shaded}
\begin{Highlighting}[]
\NormalTok{  sampleMean <-}\StringTok{ }\KeywordTok{mean}\NormalTok{(mean_simulation)}
\NormalTok{  simulationmean <-}\StringTok{ }\KeywordTok{mean}\NormalTok{(simulation)}
\NormalTok{  theoretical_mean <-}\StringTok{ }\NormalTok{(}\DecValTok{1}\OperatorTok{/}\NormalTok{lambda)}
\end{Highlighting}
\end{Shaded}

The mean of the simulations is 4.9871567\\
The theoretical mean of: \[\mu = \frac{1}{\lambda}=.5\]

\begin{figure}
\centering
\includegraphics{PeerAssignment_files/figure-latex/density-1.pdf}
\caption{Fig. 1 Simulation Distribution snd Density}
\end{figure}

Sample Variance versus Theoretical Variance: Include figures (output
from R) with titles. Highlight the variances you are comparing. Include
text that explains your understanding of the differences of the
variances.

\begin{Shaded}
\begin{Highlighting}[]
\CommentTok{# sample deviation & variance}
\NormalTok{sample_sd <-}\StringTok{ }\KeywordTok{sd}\NormalTok{(mean_simulation)}
\NormalTok{sample_var <-}\StringTok{ }\NormalTok{sample_sd}\OperatorTok{^}\DecValTok{2}
\NormalTok{simulationmean <-}\StringTok{ }\KeywordTok{mean}\NormalTok{(mean_simulation)}
\CommentTok{# theoretical deviation & variance}
\NormalTok{theoretical_sd <-}\StringTok{ }\NormalTok{(}\DecValTok{1}\OperatorTok{/}\NormalTok{lambda)}\OperatorTok{/}\KeywordTok{sqrt}\NormalTok{(n)}
\NormalTok{theoretical_var <-}\StringTok{ }\NormalTok{((}\DecValTok{1}\OperatorTok{/}\NormalTok{lambda)}\OperatorTok{*}\NormalTok{(}\DecValTok{1}\OperatorTok{/}\KeywordTok{sqrt}\NormalTok{(n)))}\OperatorTok{^}\DecValTok{2}
\NormalTok{theoretical_mean <-}\StringTok{ }\DecValTok{1}\OperatorTok{/}\NormalTok{lambda}
\end{Highlighting}
\end{Shaded}

We now compare data from the simulation to the values the CTL. CTL
predicts that the distribution of the simulation should be very close to
normal distribution.

\hypertarget{defined-as}{%
\subparagraph{Defined As}\label{defined-as}}

\[N(\mu, \sigma)\]

\hypertarget{where}{%
\subparagraph{where}\label{where}}

\[\mu=\frac{1} \lambda\]

\hypertarget{and}{%
\subparagraph{and}\label{and}}

\[\sigma =\frac {1 }{\lambda / n^2} \]

\begin{longtable}[]{@{}lrrr@{}}
\caption{Simulated vs Theoretical Values}\tabularnewline
\toprule
& Mean & Std & Variance\tabularnewline
\midrule
\endfirsthead
\toprule
& Mean & Std & Variance\tabularnewline
\midrule
\endhead
Theoretical & 5.000 & 0.791 & 0.625\tabularnewline
Simulation & 4.987 & 0.775 & 0.601\tabularnewline
\bottomrule
\end{longtable}

\hypertarget{distribution}{%
\paragraph{Distribution:}\label{distribution}}

In the figure below, we plot the histogram of the means and the density
of the means in order to note the appearant normalcy. We also note that
the distribution appears centered around.5 and is symetric. Further, the
similarity of the simulation density in blue to the theoretical density
in red highlights the proximity to normalcy of the simulation means
which exemplifies the CLT's correctneess.

\begin{figure}
\centering
\includegraphics{PeerAssignment_files/figure-latex/figs-1.pdf}
\caption{Fig. 2 Distribution Analysis}
\end{figure}

\begin{figure}
\centering
\includegraphics{PeerAssignment_files/figure-latex/figs3-1.pdf}
\caption{Fig. 3 Distribution Analysis}
\end{figure}

Q-Q Normal Plot also indicates the normal distribution. qqnorm is a
generic function the default method of which produces a normal QQ plot
of the values in y. qqline adds a line to a ``theoretical'', by default
normal, quantile-quantile plot which passes through the probs quantiles,
by default the first and third quartiles. The proximity to the xy plot
of the simulation pairs demonstrated very litte deviation from normalcy.

\hypertarget{part-2-basic-inferential-data-analysis-instructions}{%
\subsection{Part 2: Basic Inferential Data Analysis
Instructions}\label{part-2-basic-inferential-data-analysis-instructions}}

Load the ToothGrowth data and perform some basic exploratory data
analyses

\begin{Shaded}
\begin{Highlighting}[]
\KeywordTok{kable}\NormalTok{(}\KeywordTok{head}\NormalTok{(ToothGrowth))}
\end{Highlighting}
\end{Shaded}

\begin{longtable}[]{@{}rlr@{}}
\toprule
len & supp & dose\tabularnewline
\midrule
\endhead
4.2 & VC & 0.5\tabularnewline
11.5 & VC & 0.5\tabularnewline
7.3 & VC & 0.5\tabularnewline
5.8 & VC & 0.5\tabularnewline
6.4 & VC & 0.5\tabularnewline
10.0 & VC & 0.5\tabularnewline
\bottomrule
\end{longtable}

\begin{figure}
\centering
\includegraphics{PeerAssignment_files/figure-latex/ToothPlot1-1.pdf}
\caption{Fig. 4 Tooth Length}
\end{figure}

The Box Plot above seems to indicate additional tooth growth in the OJ
supplement over the VC supplement.

\begin{figure}
\centering
\includegraphics{PeerAssignment_files/figure-latex/scatter4-1.pdf}
\caption{Fig. 5 Growth by Supplement and Dose}
\end{figure}

The scatter plot show a higher level of tooth growth at lower dosages as
compared to the vitamin C. However, at the higher dosage (2mg) the
supplments seem equally effective.

\hypertarget{data-summary}{%
\subsection{Data Summary}\label{data-summary}}

\begin{Shaded}
\begin{Highlighting}[]
\KeywordTok{str}\NormalTok{(ToothGrowth)}
\end{Highlighting}
\end{Shaded}

\begin{verbatim}
## 'data.frame':    60 obs. of  3 variables:
##  $ len : num  4.2 11.5 7.3 5.8 6.4 10 11.2 11.2 5.2 7 ...
##  $ supp: Factor w/ 2 levels "OJ","VC": 2 2 2 2 2 2 2 2 2 2 ...
##  $ dose: num  0.5 0.5 0.5 0.5 0.5 0.5 0.5 0.5 0.5 0.5 ...
\end{verbatim}

\begin{Shaded}
\begin{Highlighting}[]
\KeywordTok{kable}\NormalTok{(}\KeywordTok{summary}\NormalTok{(ToothGrowth))}
\end{Highlighting}
\end{Shaded}

\begin{longtable}[]{@{}lclc@{}}
\toprule
& len & supp & dose\tabularnewline
\midrule
\endhead
& Min. : 4.20 & OJ:30 & Min. :0.500\tabularnewline
& 1st Qu.:13.07 & VC:30 & 1st Qu.:0.500\tabularnewline
& Median :19.25 & NA & Median :1.000\tabularnewline
& Mean :18.81 & NA & Mean :1.167\tabularnewline
& 3rd Qu.:25.27 & NA & 3rd Qu.:2.000\tabularnewline
& Max. :33.90 & NA & Max. :2.000\tabularnewline
\bottomrule
\end{longtable}

\hypertarget{confidence-intervals}{%
\subsection{Confidence Intervals}\label{confidence-intervals}}

\begin{Shaded}
\begin{Highlighting}[]
\CommentTok{# Subset the data for confints}
\NormalTok{ToothGrowth_oj <-}\StringTok{ }\KeywordTok{subset}\NormalTok{(ToothGrowth, supp}\OperatorTok{==}\StringTok{'OJ'}\NormalTok{)}
\NormalTok{ToothGrowth_vc <-}\StringTok{ }\KeywordTok{subset}\NormalTok{(ToothGrowth, supp}\OperatorTok{==}\StringTok{'VC'}\NormalTok{)}
\NormalTok{ConfIntTable <-}\StringTok{ }\KeywordTok{rbind}\NormalTok{(}
\KeywordTok{mean}\NormalTok{(ToothGrowth}\OperatorTok{$}\NormalTok{len) }\OperatorTok{+}\StringTok{ }\KeywordTok{c}\NormalTok{(}\OperatorTok{-}\DecValTok{1}\NormalTok{, }\DecValTok{1}\NormalTok{) }\OperatorTok{*}\StringTok{ }\FloatTok{1.96} \OperatorTok{*}\StringTok{ }\KeywordTok{sd}\NormalTok{(ToothGrowth}\OperatorTok{$}\NormalTok{len)}\OperatorTok{/}\KeywordTok{sqrt}\NormalTok{(}\KeywordTok{nrow}\NormalTok{(ToothGrowth)),}
\KeywordTok{mean}\NormalTok{(ToothGrowth_oj}\OperatorTok{$}\NormalTok{len) }\OperatorTok{+}\StringTok{ }\KeywordTok{c}\NormalTok{(}\OperatorTok{-}\DecValTok{1}\NormalTok{, }\DecValTok{1}\NormalTok{) }\OperatorTok{*}\StringTok{ }\FloatTok{1.96} \OperatorTok{*}\StringTok{ }\KeywordTok{sd}\NormalTok{(ToothGrowth_oj}\OperatorTok{$}\NormalTok{len)}\OperatorTok{/}\KeywordTok{sqrt}\NormalTok{(}\KeywordTok{nrow}\NormalTok{(ToothGrowth_oj)),}
\KeywordTok{mean}\NormalTok{(ToothGrowth_vc}\OperatorTok{$}\NormalTok{len) }\OperatorTok{+}\StringTok{ }\KeywordTok{c}\NormalTok{(}\OperatorTok{-}\DecValTok{1}\NormalTok{, }\DecValTok{1}\NormalTok{) }\OperatorTok{*}\StringTok{ }\FloatTok{1.96} \OperatorTok{*}\StringTok{ }\KeywordTok{sd}\NormalTok{(ToothGrowth_vc}\OperatorTok{$}\NormalTok{len)}\OperatorTok{/}\KeywordTok{sqrt}\NormalTok{(}\KeywordTok{nrow}\NormalTok{(ToothGrowth_vc)))}

\KeywordTok{row.names}\NormalTok{(ConfIntTable) =}\StringTok{ }\KeywordTok{c}\NormalTok{(}\StringTok{'Overall'}\NormalTok{, }\StringTok{'OJ'}\NormalTok{, }\StringTok{'VC'}\NormalTok{)}
\NormalTok{df_conf <-}\StringTok{ }\KeywordTok{as.data.frame.matrix}\NormalTok{(ConfIntTable) }
\NormalTok{df_conf }\OperatorTok\StringTok{ }
\StringTok{  }\KeywordTok{rename}\NormalTok{(}
    \StringTok{'ConfInt-Low'}\NormalTok{ =}\StringTok{ 'V1'}\NormalTok{,}
    \StringTok{'ConfInt-High'}\NormalTok{ =}\StringTok{ 'V2'}\NormalTok{)}
\end{Highlighting}
\end{Shaded}

\begin{verbatim}
##         ConfInt-Low ConfInt-High
## Overall    16.87779     20.74888
## OJ         18.29956     23.02710
## VC         14.00537     19.92129
\end{verbatim}

\hypertarget{hypothesis-testing-and-sample-analysis}{%
\subsection{Hypothesis Testing and Sample
Analysis}\label{hypothesis-testing-and-sample-analysis}}

\hypertarget{all-data}{%
\subparagraph{All Data}\label{all-data}}

\begin{Shaded}
\begin{Highlighting}[]
\CommentTok{#Run Welch Test on entire Sample}
\NormalTok{    model1 <-}\StringTok{ }\KeywordTok{t.test}\NormalTok{(len }\OperatorTok{~}\StringTok{ }\NormalTok{supp, }\DataTypeTok{data =}\NormalTok{ ToothGrowth)}
\NormalTok{    model1}
\end{Highlighting}
\end{Shaded}

\begin{verbatim}
## 
##  Welch Two Sample t-test
## 
## data:  len by supp
## t = 1.9153, df = 55.309, p-value = 0.06063
## alternative hypothesis: true difference in means is not equal to 0
## 95 percent confidence interval:
##  -0.1710156  7.5710156
## sample estimates:
## mean in group OJ mean in group VC 
##         20.66333         16.96333
\end{verbatim}

\begin{Shaded}
\begin{Highlighting}[]
\CommentTok{# Run power tests to determine power properties of our sample}
\NormalTok{    all_delta <-}\StringTok{ }\KeywordTok{power.t.test}\NormalTok{(}\DataTypeTok{n =} \KeywordTok{nrow}\NormalTok{(ToothGrowth), }\DataTypeTok{power =} \FloatTok{.90}\NormalTok{, }\DataTypeTok{sd =} \KeywordTok{sd}\NormalTok{(ToothGrowth}\OperatorTok{$}\NormalTok{len))}\OperatorTok{$}\NormalTok{delta}
\NormalTok{    all_size <-}\StringTok{  }\KeywordTok{power.t.test}\NormalTok{( }\DataTypeTok{power =} \FloatTok{.90}\NormalTok{, }\DataTypeTok{delta=}\DecValTok{3}\NormalTok{, }\DataTypeTok{sd =} \KeywordTok{sd}\NormalTok{(ToothGrowth}\OperatorTok{$}\NormalTok{len))}\OperatorTok{$}\NormalTok{n}
\NormalTok{    all_power <-}\StringTok{ }\KeywordTok{power.t.test}\NormalTok{(}\DataTypeTok{n =} \KeywordTok{nrow}\NormalTok{(ToothGrowth), }\DataTypeTok{delta =} \DecValTok{3}\NormalTok{, }\DataTypeTok{sd =} \KeywordTok{sd}\NormalTok{(ToothGrowth}\OperatorTok{$}\NormalTok{len))}\OperatorTok{$}\NormalTok{power}
\end{Highlighting}
\end{Shaded}

The p value is somewhat high for all dosages. Therefore we accept the
null H0 hypthesis that there is no difference in the means of the
supplements and also accept that there maybe some significance in
general on supplement type accross all dosages.

Power testing indicates that we could only determine a delta of 4.56 mm
with our current sample size with .9 power. If we wanted to increase the
length sensitivity to a delta of 3mm we would need a sample size of 138.
At our current sample size of 60, we can only detect a 3mm growth delta
with 0.57 power.

\hypertarget{analysis-at-.5mg-doses}{%
\subparagraph{Analysis at .5mg Doses}\label{analysis-at-.5mg-doses}}

\begin{Shaded}
\begin{Highlighting}[]
\NormalTok{five <-}\StringTok{ }\KeywordTok{subset}\NormalTok{(ToothGrowth, dose }\OperatorTok{==}\StringTok{ "0.5"}\NormalTok{)}
\NormalTok{model2 <-}\StringTok{ }\KeywordTok{t.test}\NormalTok{(len }\OperatorTok{~}\StringTok{ }\NormalTok{supp, }\DataTypeTok{data =}\NormalTok{ five)}
\NormalTok{model2}
\end{Highlighting}
\end{Shaded}

\begin{verbatim}
## 
##  Welch Two Sample t-test
## 
## data:  len by supp
## t = 3.1697, df = 14.969, p-value = 0.006359
## alternative hypothesis: true difference in means is not equal to 0
## 95 percent confidence interval:
##  1.719057 8.780943
## sample estimates:
## mean in group OJ mean in group VC 
##            13.23             7.98
\end{verbatim}

\begin{Shaded}
\begin{Highlighting}[]
\NormalTok{five_delta <-}\StringTok{ }\KeywordTok{power.t.test}\NormalTok{(}\DataTypeTok{n =} \KeywordTok{nrow}\NormalTok{(five), }\DataTypeTok{power =} \FloatTok{.90}\NormalTok{, }\DataTypeTok{sd =} \KeywordTok{sd}\NormalTok{(five}\OperatorTok{$}\NormalTok{len))}\OperatorTok{$}\NormalTok{delta}
\NormalTok{five_size <-}\StringTok{  }\KeywordTok{power.t.test}\NormalTok{(}\DataTypeTok{delta =} \DecValTok{3}\NormalTok{, }\DataTypeTok{power =} \FloatTok{.90}\NormalTok{, }\DataTypeTok{sd =} \KeywordTok{sd}\NormalTok{(five}\OperatorTok{$}\NormalTok{len))}\OperatorTok{$}\NormalTok{n}
\NormalTok{five_power <-}\StringTok{ }\KeywordTok{power.t.test}\NormalTok{(}\DataTypeTok{delta =} \DecValTok{3}\NormalTok{,}\DataTypeTok{n =} \KeywordTok{nrow}\NormalTok{(five) , }\DataTypeTok{sd =} \KeywordTok{sd}\NormalTok{(five}\OperatorTok{$}\NormalTok{len))}\OperatorTok{$}\NormalTok{power}
\end{Highlighting}
\end{Shaded}

The p value is low at the .5 mg dose level. Therefore we reject the H0
hypthesis that there is no difference in the means of the supplements
and we accept that there is some signifcance to the impact on
toothgrowth at this dose level.

Power testing indicates that we could only determine a delta of 4.73 mm
with our current sample size with .9 power. If we wanted to increase the
length sensitivity to a delta of 3mm we would need a sample size of 48.
At our current sample size of 20, we can only detect a 3mm growth delta
with 0.54 power.

\hypertarget{analysis-at-1mg-doses}{%
\subparagraph{Analysis at 1mg Doses}\label{analysis-at-1mg-doses}}

\begin{Shaded}
\begin{Highlighting}[]
\NormalTok{one <-}\StringTok{ }\KeywordTok{subset}\NormalTok{(ToothGrowth, dose }\OperatorTok{==}\StringTok{ "1"}\NormalTok{)}
\NormalTok{model3 <-}\StringTok{ }\KeywordTok{t.test}\NormalTok{(len }\OperatorTok{~}\StringTok{ }\NormalTok{supp, }\DataTypeTok{data =}\NormalTok{ one)}
\NormalTok{model3}
\end{Highlighting}
\end{Shaded}

\begin{verbatim}
## 
##  Welch Two Sample t-test
## 
## data:  len by supp
## t = 4.0328, df = 15.358, p-value = 0.001038
## alternative hypothesis: true difference in means is not equal to 0
## 95 percent confidence interval:
##  2.802148 9.057852
## sample estimates:
## mean in group OJ mean in group VC 
##            22.70            16.77
\end{verbatim}

\begin{Shaded}
\begin{Highlighting}[]
\NormalTok{one_delta <-}\StringTok{ }\KeywordTok{power.t.test}\NormalTok{(}\DataTypeTok{n =} \KeywordTok{nrow}\NormalTok{(one), }\DataTypeTok{power =} \FloatTok{.90}\NormalTok{, }\DataTypeTok{sd =} \KeywordTok{sd}\NormalTok{(one}\OperatorTok{$}\NormalTok{len))}\OperatorTok{$}\NormalTok{delta}
\NormalTok{one_size <-}\StringTok{  }\KeywordTok{power.t.test}\NormalTok{(}\DataTypeTok{delta =} \DecValTok{3}\NormalTok{, }\DataTypeTok{power =} \FloatTok{.90}\NormalTok{, }\DataTypeTok{sd =} \KeywordTok{sd}\NormalTok{(one}\OperatorTok{$}\NormalTok{len))}\OperatorTok{$}\NormalTok{n}
\NormalTok{one_power <-}\StringTok{ }\KeywordTok{power.t.test}\NormalTok{(}\DataTypeTok{delta =} \DecValTok{3}\NormalTok{, }\DataTypeTok{n =} \KeywordTok{nrow}\NormalTok{(one), }\DataTypeTok{sd =} \KeywordTok{sd}\NormalTok{(one}\OperatorTok{$}\NormalTok{len))}\OperatorTok{$}\NormalTok{power}
\end{Highlighting}
\end{Shaded}

The p value is low at the 1 mg dose level. Therefore we reject the H0
hypthesis that there is no difference in the means of the supplements
and we accept that there is some signifcance to the impact on
toothgrowth at this dose level.

Power testing indicates that we could only determine a delta of 4.65 mm
with our current sample size with .9 power. If we wanted to increase the
length sensitivity to a delta of 3mm we would need a sample size of 47.
At our current sample size of 20, we can only detect a 3mm growth delta
with 0.55 power.

\hypertarget{analysis-at-2mg-doses}{%
\subparagraph{Analysis at 2mg Doses}\label{analysis-at-2mg-doses}}

\begin{Shaded}
\begin{Highlighting}[]
\NormalTok{two <-}\StringTok{ }\KeywordTok{subset}\NormalTok{(ToothGrowth, dose }\OperatorTok{==}\StringTok{ "2"}\NormalTok{)}
\NormalTok{model4 <-}\StringTok{ }\KeywordTok{t.test}\NormalTok{(len }\OperatorTok{~}\StringTok{ }\NormalTok{supp, }\DataTypeTok{data =}\NormalTok{ two)}
\NormalTok{model4}
\end{Highlighting}
\end{Shaded}

\begin{verbatim}
## 
##  Welch Two Sample t-test
## 
## data:  len by supp
## t = -0.046136, df = 14.04, p-value = 0.9639
## alternative hypothesis: true difference in means is not equal to 0
## 95 percent confidence interval:
##  -3.79807  3.63807
## sample estimates:
## mean in group OJ mean in group VC 
##            26.06            26.14
\end{verbatim}

\begin{Shaded}
\begin{Highlighting}[]
\NormalTok{two_delta <-}\StringTok{ }\KeywordTok{power.t.test}\NormalTok{(}\DataTypeTok{n =} \KeywordTok{nrow}\NormalTok{(two), }\DataTypeTok{power =} \FloatTok{.90}\NormalTok{, }\DataTypeTok{sd =} \KeywordTok{sd}\NormalTok{(two}\OperatorTok{$}\NormalTok{len))}\OperatorTok{$}\NormalTok{delta}
\NormalTok{two_size <-}\StringTok{ }\KeywordTok{power.t.test}\NormalTok{(}\DataTypeTok{delta =} \DecValTok{3}\NormalTok{, }\DataTypeTok{power =} \FloatTok{.90}\NormalTok{, }\DataTypeTok{sd =} \KeywordTok{sd}\NormalTok{(two}\OperatorTok{$}\NormalTok{len))}\OperatorTok{$}\NormalTok{n}
\NormalTok{two_power <-}\StringTok{ }\KeywordTok{power.t.test}\NormalTok{(}\DataTypeTok{delta =} \DecValTok{3}\NormalTok{,}\DataTypeTok{n =} \KeywordTok{nrow}\NormalTok{(two), }\DataTypeTok{sd =} \KeywordTok{sd}\NormalTok{(two}\OperatorTok{$}\NormalTok{len))}\OperatorTok{$}\NormalTok{power}
\end{Highlighting}
\end{Shaded}

Zero is in the confidence interval and the p value is high. Therefore we
can not reject the null hypthesis that the means of the two supplememts
are equal at the 2mg dose for tooth growth.

Power testing indicates that we could only determine a delta of 3.97 mm
with our current sample size with .9 power. If we wanted to increase the
length sensitivity to a delta of 3mm we would need a sample size of 34.
At our current sample size of 20, we can only detect a 3mm growth delta
with 0.69 power.

\hypertarget{conclusion}{%
\subsection{Conclusion}\label{conclusion}}

The exploratory phase of the analysis indicated that there was some
possibility of correlation of tooth growth to supplement type and dose.
Both Confidence Intervals and Welch Two Sample hypothesis testing
confirmed that there is significance at the .5 mg and 1mg dose levels
and no attributable significance to the dose level by supplement type at
the 2 mg level. However, we must careful when we interpret the dosages
as the power t.test indicates that diffences in the means is just within
the required sample size. We reccomend that the analysis be conducted
with higher sample sizes in the future in order to confirm these
findings.

\end{document}
